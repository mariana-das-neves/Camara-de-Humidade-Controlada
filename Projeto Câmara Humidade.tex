\documentclass[journal]{IEEEtran}
\usepackage{graphicx, hyperref}
\usepackage{subcaption,lipsum}
\usepackage{dblfloatfix}
\usepackage{float}
\usepackage{multicol}
\usepackage[toc,page]{appendix}

\ifCLASSINFOpdf

\else
  % or other class option (dvipsone, dvipdf, if not using dvips). graphicx
\fi
% correct bad hyphenation here
\hyphenation{op-tical net-works semi-conduc-tor}

\begin{document}

\title{Câmara de Humidade Controlada}

\author{João Raúl Vidal Carvalho up202004346 \\ Mariana Gonçalves Neves up202008785 \\ Sofia Marques Duarte up202007789 \\ Departamento de Física da Faculdade de Ciências da Universidade do Porto}

% The paper headers
\markboth{PROJETO INTEGRADOR}%
{Shell \MakeLowercase{\textit{et al.}}: Bare Demo of IEEEtran.cls for IEEE Journals}

\maketitle

\section{Abstract}

O objetivo principal é a monitorização e controlo da humidade relativa do ar de uma câmara fechada utilizando uma interface em LabView. Através desta interface deverá ser possível fazer rampas de aumento e diminuição de humidade relativa ou manter a humidade relativa do ar em valores estáveis. 




\section{Fase de Conceção}

- OBJETIVOS
- IDENTIFICAR O PROBLEMA A RESOLVER
- DEFINIR CONCEITO


Os objetivos de trabalho incluem a monitorização e controlo da humidade relativa do ar dentro de uma câmara fechada recorrendo a uma interface $LabVIEW$. Este controlo inclui a estabilização da humidade recorrendo ao seu aumento ou decréscimo.





\section{Descrição dos Casos de Uso}

- DESENVOLVER CASOS DE ESTUDO DETALHADOS
- DESCREVER COMO DIFERENTES UTILIZADORES IRÃO INTERAGIR COM O SISTEMA
- COMPREENDER OS CENÁRIOS EM QUE IRÃO OPERAR
- ESBOÇAR AÇÕES ESPECÍFICAS DOS CASOS DE ESTUDO 


Neste projeto teremos dois modos a operar: um manual e outro automático.
No modo manual, o utilizador poderá interagir com o sistema, controlando o funcionamento do humidificador, desumidificador e ventoinhas.

No modo automático, o utilizador coloca o valor de humidade que pretende que a câmara atinja. Caso seja superior ao valor de humidade detetado pelos sensores, é ligado o humidificador; caso contrário, será ligado o desumidificador. 
Será dada uma margem de erro, para evitar o bouncing, e, assim que o valor ideal for atingido, os componentes serão desligados.







\section{Especificação de Requisitos}

-> REQUISITOS FUNCIONAIS: O QUE O SISTEMA DEVE FAZER
-> REQUISITOS NÃO FUNCIONAIS: COMO O SISTEMA DEVE ATUAR SOB VÁRIAS CONDIÇÕES

// Estas 4 secções criadas acho que se poderiam juntar numa só. Acho que só têm importância para projetar o trabalho e não para o relatório





\section{Design do Sistema e Especificação de Componentes}

-> DESENHAR ARQUITETURA DO SISTEMA
-> INSTRUMENTOS E MATERIAIS NECESSÁRIOS

\subsection{Material}

%O material base para este projeto inclui sensores de humidade e temperatura, ventoinhas, um humidificador de ultrassons comercial, um desumidificador comercial, relés para ativação de vários componentes, microcontroladores com interface LabView e eletrónica diversa.

%esta parte de cima é redundante com o que está embaixo, pode-se tirar, não?


Para este projeto foram precisos os seguintes materiais:
\begin{itemize}
    \item Microcontrolador Arduino UNO;
    \item 2 sensores de temperatura e humidade DHT22;
    \item Humidificador de ultrassons comercial [USB Colorful Humidifier];
    \item Desumidificador comercial;
    \item Ventoinhas;
    \item Breadboard e cabos de ligação;
    \item Relés para ativação de vários componentes;
    \item FETs;
    \item Resistências
\end{itemize}

A validação deste projeto pode ser efetuada usando uma câmara fechada existente nas instalações do INESC TEC. 

\subsection{Procedimento}

Serão ligados relés aos diferentes componentes para controlar o seu funcionamento. Ao desumidificador e ventoinhas será ligado um, enquanto que ao humidificador, devido aos seus diferentes modos, serão ligados dois relés:

Um que funcionará como ligação à fonte de alimentação e um outro que funcionará como botão para controlar os modos. 

Um toque no botão e é ativado o modo contínuo;
Clicando outra vez, é ativado o modo intermitente;
Pressionando durante 3 segundos, é ativado o brilho;
Pressionado outros 3 segundos, é ativado o modo de cores.



%meter foto do circuito



%será dito mais à frente a partir do código Arduino:
%Desumificador = 4
%Ventoinhas = 12 e 8
%Sensores = 6 e 7
%Humificadores = 2 alimentação, 3 switch



\section{Implementação e Integração}

%CONSTRUÇÃO


%CODIFICAÇÃO;

\subsection{Código Arduino}

Este foi o código utilizado para definir os pinos do Arduino UNO associados aos componentes. Os sensores foram ligados aos pinos 6 e 7; as ventoinhas aos pinos 8 e 12;
o desumidificador ao 4 e os dois relés do humidificador aos pinos 2 e 3.


 \begin{figure}[H]
    \centering
    \includegraphics[width=0.80\textwidth]{Arduino1.jpeg}
    \label{fig:1}
\end{figure}

Aqui define-se o código para ligar e desligar as ventoinhas, bem como desumidificador, digitando os respetivos caracteres no Serial Monitor.

De notar que as duas ventoinhas não funcionam de modo independente uma da outra. 


 \begin{figure}[H]
    \centering
    \includegraphics[width=0.80\textwidth]{Arduino2.jpeg}
    \caption{\label{fig:2} Código no Arduino IDE}
    \label{fig:2}
\end{figure}



Esta parte está destinada ao controlo do humidificador que, tendo 3 modos, terá 3 funções: modo contínuo, modo pulsado e modo desligado.

 \begin{figure}[H]
    \centering
    \includegraphics[width=0.80\textwidth]{Arduino3.jpeg}
    \caption{\label{fig:3} Código no Arduino IDE}
    \label{fig:3}
\end{figure}


Este foi o código utilizado para podermos detetar os valores de temperatura e humidade dos 2 sensores DHT22. Os respetivos valores aparecerão no Serial Monitor do Arduino IDE.

 \begin{figure}[H]
    \centering
    \includegraphics[width=0.80\textwidth]{Arduino4.jpeg}
    \caption{\label{fig:4} Código no Arduino IDE}
    \label{fig:4}
\end{figure}




\subsection{LabVIEW}

Para conseguir ler os valores de temperatura e humidade dos 2 sensores em LabVIEW:

 \begin{figure}[H]
    \centering
    \includegraphics[width=.9\linewidth]{labview1.jpg}
    \caption{\label{fig:1} Interface LabVIEW}
    \label{fig:3}
\end{figure}

Vamos tomar partido das médias da temperatura e humidade para prosseguir o trabalho.

% MONTAGEM







\section{Testes e Otimização}

-> REALIZAÇÃO DE TESTES EXAUSTIVOS
-> IDENTIFICAÇÃO DE DISCREPÂNCIAS E DEFEITOS
-> IDENTIFICAÇÃO DE MELHORIAS
-> OTIMIZAÇÃO






\section{Revisão do Projeto e Encerramento}






 
\section{Resultados e conclusão}







\section{Anexos}









\newpage
\section{Appendices}






\subsection{Uncertainties}

If $x$ is a function of other variables, for example: $x = f(y,z,w)$, Its uncertainty is determined by expression \ref{equa:uncertainty}:

\begin{equation}
    \Delta x = x \cdot \sqrt{(\frac{\Delta y}{y})^2+(\frac{\Delta z}{z})^2+(\frac{\Delta w}{w})^2}
    \label{equa:uncertainty}
\end{equation}

\bibliographystyle{plain}
\bibliography{bib}





\end{document}


